%~ xelatex -shell-escape grrVerkefni3.tex

\documentclass[a4paper,oneside]{article}
\usepackage{a4wide}
\usepackage[icelandic]{babel}
\usepackage{fontspec}
\usepackage{xunicode}
\usepackage{graphicx}
\usepackage{enumerate}

\usepackage{mathtools} % := from \coloneqq
\usepackage{units} % falleg 1/3 brot
\usepackage{amsmath} % \begin{align} (also in \{mathtools})
\usepackage{amsthm} % proof
\usepackage{amssymb} % blacksquare og mathbb
 \renewcommand{\qedsymbol}{$\blacksquare$}

\usepackage{minted}
\usemintedstyle{perldoc} % default, bw, perldoc
\setlength\partopsep{-\topsep} % eyðir út bili að ofan og neðan kóða
%\inputminted[ options ]{ language }{ filename }

\usepackage{fancyhdr}
\pagestyle{fancy}
\usepackage{hyperref}

\lhead{Greining reiknirita}
\chead{Verkefni 3}
\rhead{Háskóli Íslands}

\title{
    Greining reiknirita:\ Verkefni 3
    \\\small{Kennari: Páll Melsted}
}
\author{Bjarki Geir Benediktsson \and  Bjarni Jens Kristinsson \and Tandri Gauksson}
\date{\small{Skil:\ 6. apríl 2014}}


\begin{document}
\maketitle

\section{Lýsing á reikniritinu}
Við byggðum lausn okkar á klasanum \texttt{KruskalMST.java}\footnote{\url{http://algs4.cs.princeton.edu/43mst/KruskalMST.java.html}} sem er að finna á heimasíðu bókarinnar \emph{Algorithms} eftir Sedgewick og Wayne. Smiður klasans tekur inn óstefnt vigtað net $G=(V, \ E, \ W)$ og notar reiknirit Kruskals til að finna minnsta spanntré $mst$ og vigt þess $w$.

Við skrifuðum nýja aðferð \texttt{secondMSTweight( EdgeWeightedGraph G, Edge e )} sem að finnur vigt minnsta spanntré netsins $G_e=(V, \ E \smallsetminus \{ e \}, \ W)$. Í \texttt{keyrsluforrit.java} finnum við fyrst $mst$ og prentum vigt þess. Síðan tökum við hvern legg $e \in mst$ (raðað í vaxandi röð eftir fyrra hniti $e$) og finnum minnsta spanntré netsins $G_e$ og prentum legginn $e$ og vigt þess $w_e$ á staðalúttak. 

Við útfærðum \texttt{secondMSTweight( EdgeWeightedGraph G, Edge e )} svipað og smiðinn fyrir \texttt{KruskalMST}. Í staðinn fyrir að fara línulega í gegn um alla leggi $V$ og kanna hvor t þeir tengja saman samhengisþætti í skóginum fyrir netið þá byggjum við upp \texttt{UF} hlutinn með öllum leggjunum úr $E \smallsetminus \{ e \}$. Þetta skilur okkur eftir með tvo samhengisþætti, eða tvö tré í skóginum, og þá förum við í vaxandi röð í gegn um $E$ og könnum hvort að hnúturinn sé $e$ og ef ekki hvort hann tengir saman trén tvö. Ef hann gerir það þá erum við komin með minnsta spannandi hluttré fyrir $G_e$. \\

\noindent
\texttt{KruskalMST.java} notast við eftirfarandi klasa frá sömu höfundum (okkar breytingar eru taldar upp á eftir hverjum og einum klasa):
\begin{itemize}
    \item \texttt{EdgeWeightedGraph.java}\footnote{\url{http://algs4.cs.princeton.edu/43mst/EdgeWeightedGraph.java.html}}
        \begin{itemize}
            \item Breyttum öllum línum sem búa til nýjan hlut af taginu \texttt{Edge} til að nota heiltöluvigt í stað fleytitölu (sjá athugasemdir við \texttt{Edge.java})
        \end{itemize}
    \item \texttt{Queue.java}\footnote{\url{http://algs4.cs.princeton.edu/43mst/Queue.java.html}}
    \item \texttt{Edge.java}\footnote{\url{http://algs4.cs.princeton.edu/43mst/Edge.java.html}}
        \begin{itemize}
            \item Breyttum vigtinni yfir í að vera heiltala í stað fleytitölu
            \item toString() aðferðin gefur örlítið öðru vísi strengjaframsetningu á hnútunum 
            \item Bættum við toString2() til að prenta út hnút og vigt eins og er í .out skránum sem Páll gefur upp á heimasíðunni
            \item Bættum við klasann lexiCompare sem ber saman hnúta eftir orðabókaröðun
        \end{itemize}
    \item \texttt{MinPQ.java}\footnote{\url{http://algs4.cs.princeton.edu/24pq/MinPQ.java.html}}
    \item \texttt{UF.java}\footnote{\url{http://algs4.cs.princeton.edu/15uf/UF.java.html}}
    \item \texttt{Bag.java}\footnote{\url{http://algs4.cs.princeton.edu/13stacks/Bag.java.html}}
    \item \texttt{Stack.java}\footnote{\url{http://algs4.cs.princeton.edu/13stacks/Stack.java.html}}
\end{itemize}


\section{Tímaflækja verkefnisins}
Hér er $G$ upphaflega netið sem \texttt{KruskalMST} hluturinn er smíðaður með, $V$ fjöldi hnúta þess og $E$ fjöldi leggja.

Að búa til hlut af taginu \texttt{KruskalMST} tekur $O(E \log V)$ tíma. Hluturinn er minnsta spanntré $G$ og inniheldur $V-1$ leggi. Við köllum einu sinni á \texttt{secondMSTweight} fyrir hvern legg

\pagebreak
\section{Java kóði}
\subsection{KruskalMST.java}
\inputminted[]{java}{KruskalMST.java}

\subsection{keyrsluforrit.java}
\inputminted[]{java}{keyrsluforrit.java}

\end{document}